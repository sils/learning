\documentclass[10pt,a4paper]{article}
\usepackage[left=1.5cm,right=1.5cm,top=1.5cm,bottom=1.5cm]{geometry}
\usepackage[utf8]{inputenc}
\usepackage[german]{babel}
\usepackage{amsmath}
\usepackage{amsfonts}
\usepackage{amssymb}
\usepackage{amsthm}
\usepackage{graphicx}

\newenvironment{packed_enum}{
\begin{enumerate}
  \setlength{\itemsep}{1pt}
  \setlength{\parskip}{0pt}
  \setlength{\parsep}{0pt}
}{\end{enumerate}}
\newcommand{\abs}[1]{\ensuremath{\left\vert#1\right\vert}}
\begin{document}
\section{Funktionen}
\subsection{Rect}
\[ a\cdot rect\left(\frac{t-t_0}{2T}\right) = \begin{cases}
a & |t-t_0| < T \\
a\over 2 & |t-t_0| = T \\
0 & sonst \\
\end{cases} \]

\subsection{step}
\[b\cdot\epsilon(a\cdot (t-t_0)) = b\cdot\epsilon(t-t_0) =
\begin{cases}
0 & t < t_0 \\
b \over 2 & t=t_0\\
b & t > t_0\\
\end{cases}\]

\subsection{tri}
\[\Lambda(t) =
\begin{cases}
0& |t|>1\\
1-t & 0 \leq t \leq 1\\
1+t & -1 \leq t \leq 0\\
\end{cases} \]

\subsection{dirac Impuls}
\subsubsection{Normal}
\[ \delta(a\cdot (t-t_0)) = \frac{1}{|a|} \delta(t-t_0) =
\begin{cases}
\neq 0 & t=t_0\\
0 & t \neq t_0\\
\end{cases}\,\,\,\,\,\,,\,\,\,\,\,\,\, \delta(-t)=\delta(t)
\,\,\,\,\,\,,\,\,\,\,\,\,\,\int\limits_{-\infty}^\infty \delta(t) dt = 1 \]

Sifting bzw. Siebung: $\delta(t-t_0)\cdot f(t) = f(t_0)\cdot \delta(t-t_0) \,\,\,\,\,\,,\,\,\,\,\,\,\,
\int\limits_{-\infty}^\infty \delta(t-t_0)\cdot f(t) dt  =
\begin{cases}
f(t) & t = t_0 \\
0 & sonst \\
\end{cases}$

\subsubsection{Impulssequenz}
\[
w_T(t) = \sum\limits_{k=-\infty}^\infty \delta(t-kT)
\]

\subsection{si}
\[
si(t) = \frac{sin(t)}{t}\,\,\,\,\,\,,\,\,\,\,\,\,\, si(0) = 1\,\,\,\,\,\,,\,\,\,\,\,\,\, \int\limits_{-\infty}^\infty si(t) dt= \int\limits_{-\infty}^\infty si^2(t) dt = \pi
\]

\subsection{ramp}
\[ \rho(t) = \epsilon(t)\cdot t \]

\subsection{sgn}
\[ sgn(x) = \frac{x}{|x|} = 2 \epsilon(t) -1 \,\,\,\,\,\,,\,\,\,\,\,\,\, |x(t)| = sgn(x(t)) \cdot x(t) \]

\section{Symmetrien}
\[x(t) = x_e(t) + x_o(t)\]
\subsection{Gerade}
\[ x_e(t) = x_e(-t) = \frac{x(t)+x(-t)}{2}
\]
\subsection{Ungerade}
\[ x_o(t) = -x_o(-t) = \frac{x(t)-x(-t)}{2}
\]

\subsection{Komplex}
$x_R$ sei der Realteil, $x_I$ sei der Imaginaerteil.

Konjugiert gerades Signal: $\,\,\,\,\,\, x(t) = \,\,\,\,\,x^*(-t) \Rightarrow x_R(t) = x_e(t) \mbox{ und } j\cdot x_I(t) = x_o(t)$

Konjugiert ungerades Signal: $x(t) = -x^*(-t) \Rightarrow x_R(t) = x_o(t) \mbox{ und } j\cdot x_I(t) = x_e(t)$

\section{Leistung und Energie von Signalen}
Fuer eine Funktion $x(t)$.
\subsection{Energie}
Energiesignale haben keine ($0$) Leistung.
\[
E_x = \int\limits_{-\infty}^\infty \left| x(t) \right|^2 dt = r_{xx}^E (0)
\]
\subsection{Leistung}
Leistungssignale haben unendliche Energie.
\[
P_x = \lim\limits_{T \rightarrow \infty} \frac{1}{2T} \int\limits_{-T}^T \left| x(t) \right|^2 dt = r_{xx}^P(0)
\]

\section{Abtastung}
\subsection{Abtastbarkeit}
Ein Signal kann nur perfekt abtastbar sein wenn die Funktion im Frequenzbereich bandbegrenzt ist.

perfekt abtastbar $\Leftrightarrow f_s = \frac{1}{T_s} \geq 2B$

\subsection{Nu aber zur Sache}
Abtastfrequenz $f_s$. Dann low-pass filter vorweg: $rect\left(\frac{f}{f_s}\right)$ ($f_g = f_s \over 2$).

\section{Laplace}
$H(s) = \frac{Y(s)}{X(s)}$

\section{Korrelation}
\[
r_{xy}^E(\tau) = \langle x(t), y(t+\tau) \rangle
\]
\[
r_{xy}^P(\tau) = \lim\limits_{T \rightarrow \infty} \frac{1}{2T} \int\limits_{-T}^T x^*(t)y(t+\tau) dt
\]

\subsection{Autocorrelation}
$r_{xx}$.

\section{Verschiedenes}
\subsection{Trigonometrie}
\[sin(-t) = -sin(t) \,\,\,\,\,\,,\,\,\,\,\,\,\, cos(-t) = cos(t) \,\,\,\,\,\,,\,\,\,\,\,\,\, \int\limits_{- \frac{T}{2}}^{\frac{T}{2}} sin^2(t) dt = \frac{1}{2}
\,\,\,\,\,\,,\,\,\,\,\,\,\,
2sin(x) cos(x) = sin(2x)\,\,\,\,\,\,,\,\,\,\,\,\,\,
2 sin^2(x) = 1-cos(2x)\]

\subsection{Orthogonalitaet}
Zwei Signale $x(t)$ und $y(t)$ sind genau dann Orthogonal wenn: $\langle x(t), y(t)\rangle =
\int\limits_{-\infty}^\infty x^*(t)y(t) dt = \sum\limits_{k=-\infty}^\infty x^*(k) y(k) = 0$

\subsection{LTI System}

Linear Time Invariant

\begin{packed_enum}
\item Ist linear
\item Konstant ueber Zeitachse
\end{packed_enum}

Output ist faltung von input und impulsantwort.h


\end{document}