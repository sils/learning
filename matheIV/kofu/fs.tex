\documentclass[10pt,a4paper]{article}
\usepackage[left=1.5cm,right=1.5cm,top=1.5cm,bottom=1.5cm]{geometry}
\usepackage[utf8]{inputenc}
\usepackage[german]{babel}
\usepackage{amsmath}
\usepackage{amsfonts}
\usepackage{amssymb}
\usepackage{amsthm}
\usepackage{graphicx}

\newenvironment{packed_enum}{
\begin{enumerate}
  \setlength{\itemsep}{1pt}
  \setlength{\parskip}{0pt}
  \setlength{\parsep}{0pt}
}{\end{enumerate}}
\newcommand{\abs}[1]{\ensuremath{\left\vert#1\right\vert}}
\begin{document}

KoFu

\section{Reihen}
\subsection{Taylorreihen}
Siehe Aufg. 16 Anleitung
Umformen, Geometrische Reihe, Integral reinziehen, Grenzen einsetzen nicht vergessen!


\subsection{Laurentreihen}
\subsubsection{Hauptteil}
\[
\frac{a}{z-z_k}
\]
\subsubsection{Nebenteil}
\[
\frac{z-z_k}{a}
\]

\section{Kreisintegrale}
\begin{enumerate}
\item Singularitaet liegt ausserhalb des Kreises -> integral ist null
\item Hebbare Singularitaet vorhanden -> integral ist null
\item Nichthebbare Singularitaet im Kreis:
\begin{enumerate}
\item Cauchysche Itegralformel
\item Verallg. Cauchysche Integralformel
\end{enumerate}
\end{enumerate}

\section{•}
\subsection{Holomorphie}
Eine Abbildung $f$ heisst holomorph wenn sie in jedem Punkt $z_0$ komplex differenzierbar ist.

Cauchy-Riemannsche DGL:
\[
\frac{\partial u}{\partial x} = \frac{\partial v}{\partial y}
\]\[
\frac{\partial u}{\partial y} = - \frac{\partial v}{\partial x}
\]

\subsection{Harmonie}
Der Realteil einer Funktion ist harmonisch wenn die Funktion holomorph ist.

Eine Funktion $f=u + iv$ ist harmonisch wenn gilt: $f_{xx} + f_{yy} = 0$


\end{document}
