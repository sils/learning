\documentclass[10pt,a4paper]{article}
\usepackage[left=1.5cm,right=1.5cm,top=1.5cm,bottom=1.5cm]{geometry}
\usepackage[utf8]{inputenc}
\usepackage[german]{babel}
\usepackage{amsmath}
\usepackage{amsfonts}
\usepackage{amssymb}
\usepackage{amsthm}
\usepackage{graphicx}

\newenvironment{packed_enum}{
\begin{enumerate}
  \setlength{\itemsep}{1pt}
  \setlength{\parskip}{0pt}
  \setlength{\parsep}{0pt}
}{\end{enumerate}}
\newcommand{\abs}[1]{\ensuremath{\left\vert#1\right\vert}}
\begin{document}

\part*{Komplexe Funktionen}

\section{Vorletzte Seite schwarze FS kopieren}

\section{Möbius Transformation}
\subsection{Bestimmen}
Transformation ist gegeben mit $T(z_1)=t_1$, $T(z_2)=t_2$ und $T(z_3)=t_3$.
\[w = \frac{
t_1(z-z_2)(z_3-z_1)(t_3-t_2) - t_2(z-z_1)(z_3-z_2)(t_3-t_1)
}{
\,\,\,\,\,(z-z_2)(z_3-z_1)(t_3-t_2) - \,\,\,\,\,(z-z_1)(z_3-z_2)(t_3-t_1)
}\]
\begin{enumerate}
\item Zwei Punkte bestimmen die symmetrisch zu beiden Kreisen liegen: $(z_1 - z_0)(\overline{z_2} - \overline{z_0}) = R^2$ fuer alle Gleichungen $|z - z_0| = R$ aufstellen, Gleichungssystem loesen
\item Transformation bestimmen: $T(a) = 0 \Rightarrow T_1(x) = x - a$, $T(b) = \infty \Rightarrow T_2(x) = \frac{x-a}{x-b}$, $T(c) = d \Rightarrow T(x) = \frac{x-a}{x-b} \cdot e = d$ erfuellen
\end{enumerate}
\subsection{Kreise}
Die Punkte $z_0$, $z_1$, $z_2$ und $z_3$ liegen auf einem verallgemeinerten Kreis wenn:
$\frac{z_0-z_1}{z_0-z_2} : \frac{z_3-z_1}{z_3-z_2} \in R$

\section{Reihen}
\subsection{Taylorreihen}
Siehe Aufg. 16 Anleitung
Umformen, Geometrische Reihe, Integral reinziehen, Grenzen einsetzen nicht vergessen!


\subsection{Laurentreihen}
\begin{enumerate}
\item Bestimme Singularitaeten
\item Hebe Hebbare Singularitaeten
\item In Summe aus Hauptteilen ($ \frac{a}{(z-z_k)^k}$) und Nebenteilen ($ \frac{z-z_k}{a}$) umformen.
\end{enumerate}

\subsubsection{Singularitäten}
Ringe $\rightarrow$ Pole
\begin{enumerate}
 \item Hebbar
 \item n-facher Pol
 \item Wesentliche Singularität
\end{enumerate}

\subsection{Residuen}
Berechnung ohne Laurent:
\begin{enumerate}
\item Polstelle 1. Ordnung: $Res(f,a) = \lim\limits_{z\rightarrow a} (z-a)f(z)$
\item Polstelle n. Ordning: $Res(f,a) = \frac{1}{(n-1)!} \lim\limits_{z\rightarrow a} \left(\frac{d}{dz}\right)^{n-1} (z-a)^n f(z)$
\item $f(z)=\frac{g(z)}{h(z)} \,\,\,\, g(a) \neq 0, h(a) = 0, h'(a)\neq 0 \Rightarrow Res(f,a)=\frac{g(a)}{h(a)}$
\end{enumerate}

Mit Laurent:
$z_k$ seien die Singularitaeten, $Res(f, a_k)$ die dazugehoerigen Residuen wobei der $1$-te Hauptteil wie folgt aussieht: $\frac{Res(f, a_k)}{z-z_0}$
\begin{itemize}
\item $\oint\limits_{K} f(z) dz = 2 \pi i \sum\limits_{z_k \neq 0} Res(f, a_k)$ wobei $Res(f, a_k)$ im Kreis
\item $\oint\limits_{-\infty}^{\infty} f(z) dz = 2 \pi i \sum\limits_{z_k \neq 0} Res(f, a_k)$ wobei $Im(Res(f, a_k)) \geq 0$
\item $\oint\limits_{0}^{\infty} f(z) dz = \pi i \sum\limits_{z_k \neq 0} Res(f, a_k)$
\end{itemize}

\subsection{Partialbruchzerlegung}
\begin{enumerate}
\item Residuen berechnen
\item Fuer alle Residuen: $f = \sum\limits_a \frac{Res(f, a)}{z-a}$
\end{enumerate}

\section{Kreisintegrale}
\begin{enumerate}
 \item Singularitaet liegt ausserhalb des Kreises - integral ist null
 \item Hebbare Singularitaet vorhanden - integral ist null
 \item Nichthebbare Singularitaet im Kreis:
 \begin{enumerate}
  \item Cauchysche Itegralformel
  \item Verallg. Cauchysche Integralformel
 \end{enumerate}
\end{enumerate}

\section{Funktionseigenschaften}
\subsection{Holomorphie}
Eine Abbildung $f$ heisst holomorph wenn sie in jedem Punkt $z_0$ komplex differenzierbar ist.

Cauchy-Riemannsche DGL:
\[
\frac{\partial u}{\partial x} = \frac{\partial v}{\partial y}
\,\,\,\,\,\,\,\,\,\,\,\,\,\,\,\,\,\,\,\,\,\,\,\,
\frac{\partial u}{\partial y} = - \frac{\partial v}{\partial x}
\]

Holomorphie impliziert harmonie.

\subsection{Harmonie}
Der Realteil einer Funktion ist harmonisch wenn die Funktion holomorph ist.

Eine Funktion $f=u + iv$ ist harmonisch wenn gilt: $f_{xx} + f_{yy} = 0$


\end{document}
