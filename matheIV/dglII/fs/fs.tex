\documentclass[10pt,a4paper]{article}
\usepackage[left=1.5cm,right=1.5cm,top=1.5cm,bottom=1.5cm]{geometry}
\usepackage[utf8]{inputenc}
\usepackage[german]{babel}
\usepackage{amsmath}
\usepackage{amsfonts}
\usepackage{amssymb}
\usepackage{amsthm}
\newenvironment{packed_enum}{
\begin{enumerate}
  \setlength{\itemsep}{1pt}
  \setlength{\parskip}{0pt}
  \setlength{\parsep}{0pt}
}{\end{enumerate}}
\newcommand{\abs}[1]{\ensuremath{\left\vert#1\right\vert}}
\begin{document}

\section{Homogene lineare partielle Differentialgleichungen 1. Ordnung}
\begin{packed_enum}
\item Umstellen in Normalform: $\sum\limits_{i = 1}^n a_i(\vec{x}) \cdot u_i = 0$, weitere Beispiele für $n=2$.
\item Charakteristische DGLs sind zu lösen: $\dot{x}(t) = a_1(x(t), y(t)); \dot{y}(t) = a_2(x(t), y(t))$
\begin{packed_enum}
\item Dabei darf man alle Gleichungen z.B. durch eine andere teilen
\item Hat man dies getan, erhält man z.B. $\dot{x}(t)=1$ und somit $x(t)=t+c$ und $c=0$ ist zulässig, also $x=t$
\end{packed_enum}
\item Eine der Gleichungen durch Variablen der anderen darstellen: $y = \psi(x, C)$, nach $C(\vec{c})$ umstellen $C(\vec{c}) = \phi(x,y)$
\item $\Phi(\phi(x,y))$ mit beliebigem $\Phi$ aus $C^1$ ist die Lösung
\item Allgemein: $\Phi(\phi_1({x}), ... , \phi_{n-1}(\vec{x}))$
\end{packed_enum}

\end{document}
