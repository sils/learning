\documentclass[10pt,a4paper,oneside]{report}
\usepackage[utf8]{inputenc}
\usepackage{amsmath}
\usepackage{amsfonts}
\usepackage{amssymb}
\usepackage{hyperref}
\usepackage[left=3.5cm,right=3.5cm,top=3.5cm,bottom=3.5cm]{geometry}
\setlength{\parskip}{3mm}
\setlength{\parindent}{0cm}
\title{Database Systems - Formulas}
\author{Lasse Schuirmann}
\newcommand*{\blankpage}{
  \vspace*{\fill}
  \begin{flushright}
  \tiny THIS PAGE INTENTIONALLY LEFT BLANK.
  \end{flushright}
  \pagebreak
}
\begin{document}
\maketitle

\pagebreak

\blankpage

\section*{Schlüssel}

\subsection*{Superschlüssel}

Ein Superschlüssel definiert implizit alle anderen Attribute der Relation.

\subsection*{Schlüsselkandidat}

Ein Schlüsselkandidat ist ein minimaler Superschlüssel.

\subsection*{Primärschlüssel}

Ein Primärschlüssel ist der ausgewählte Schlüsselkandidat.

\section*{Komische Symbole}

$\alpha \rightarrow \beta \Leftrightarrow \alpha$ bestimmt eindeutig/ist Superschlüssel für $\beta$

$\alpha \dot{\rightarrow} \beta \Leftrightarrow \alpha$ ist Schlüsselkandidat für $\beta$

\section*{Normalformen}

\subsection*{1. NF}

Attribute sind atomar.

\subsection*{2. NF}

Attribute sind nicht abhängig von einer echten Teilmenge eines Schüsselkandidaten.

\subsection*{3. NF}

Attribute sind ausschliesslich abhängig von dem Primärschluessel.
\end{document}
