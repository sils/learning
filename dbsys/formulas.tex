\documentclass[10pt,a4paper,oneside]{report}
\usepackage[utf8]{inputenc}
\usepackage{amsmath}
\usepackage{amsfonts}
\usepackage{amssymb}
\usepackage{hyperref}
\usepackage[left=3.5cm,right=3.5cm,top=3.5cm,bottom=3.5cm]{geometry}
\setlength{\parskip}{3mm}
\setlength{\parindent}{0cm}
\title{Database Systems - Formulas}
\author{Lasse Schuirmann}
\newcommand*{\blankpage}{
  \vspace*{\fill}
  \begin{flushright}
  \tiny THIS PAGE INTENTIONALLY LEFT BLANK.
  \end{flushright}
  \pagebreak
}
\begin{document}
\maketitle

\pagebreak

\blankpage

\section*{ER Diagramme}

$kunst \overset{N}{-} sein \overset{M}{=} Ausstellung$

Lesen wie:

Kunst KANN in M Ausstellungen sein.

Eine Ausstellung MUSS (1-) N Kunstgegenstände enthalten.

\section*{Relationales Schema}

Object: \{[ \underline{Primärschlüssel: Typ}, Andere Schluessel: Andere Typen ]\}

Prädikat: \{[ PrimärschlüsselVonObjekt1: Typ, PrimärschlüsselVonObjekt2: Typ, Eigenschaften: Typ ]\}

\section*{Schlüssel}

\subsection*{Superschlüssel}

Ein Superschlüssel definiert implizit alle anderen Attribute der Relation.

\subsection*{Schlüsselkandidat}

Ein Schlüsselkandidat ist ein minimaler Superschlüssel.

\subsection*{Primärschlüssel}

Ein Primärschlüssel ist der ausgewählte Schlüsselkandidat.

\section*{Komische Symbole}

\subsection*{Attributrelationen}

$\alpha \rightarrow \beta \Leftrightarrow \alpha$ bestimmt eindeutig/ist Superschlüssel für $\beta$

$\alpha \dot{\rightarrow} \beta \Leftrightarrow \alpha$ ist Schlüsselkandidat für $\beta$

\subsection*{Queries}

$\sigma_c(R)$ ist eine Anfrage auf die Datenbank $R$ mit der Bedingung $c$. (SELECT ... FROM $R$ WHERE $c$)

$\pi_L(R)$ Projektion (SELECT something AS else FROM ...)

$\cup$ Vereinigung

$\cap$ Durchschnitt

$-$ Differenz (SELECT $*$ FROM $T1$ WHERE $ID$ NOT IN (SELECT $ID$ FROM $T2$))

$R \times L$ Kartesisches Produkt (SELECT ... FROM $R, L$)

$\bowtie$ Verbund/Join (JOIN ... ON)

$\gamma$ Gruppierung (GROUP BY)

$\delta$ Duplikationselimination (SELECT DISTINCT)

$\tau$ Sortieren (ORDER BY ... [ASC/DESC])

\subsection*{Volcano-Iterator-Modell}

Wie python Iteratoren.

Iteratoren haben dann: open(), next(), close()

Dann:

\begin{enumerate}
\item open() fuer alle operatoren durchfuehren
\item next() fuer alle operatoren durchfuehren, ergebnisse durchreichen
\item letzten Schritt wiederholen bis next() eof/null zurueckgibt
\item close() fuer alle operatoren durchfuehren
\end{enumerate}

\section*{Normalformen}

\subsection*{1. NF}

Attribute sind atomar.

\subsection*{2. NF}

Attribute sind nicht abhängig von einer echten Teilmenge eines Schüsselkandidaten.

\subsection*{3. NF}

Attribute sind ausschliesslich abhängig von dem Primärschluessel.

\section*{RAID}

\subsection*{RAID 0}

Reissverschlussverfahren um Zugriffszeiten zu optimieren.

\subsection*{RAID 1}

Spiegelung - komplettbackup. Kein Performancegewinn.

\subsection*{RAID 5}

Eine Paritaetsplatte, kompensiert Ausfall einer Platte. Performancegewinn durch aufteilung.

\section*{B+ Baum}

Ähnlich eines binären Baums, kann zur Suche benutzt werden. Ein Baum mit der Tiefe $d$ hat an jedem Knoten $d$ bis $2*d$ Unterknoten. An dem Wurzelknoten können weniger Unterknoten vorhanden sein.

Ein geclusterter B+ Baum ist im Speicher ebenso sortiert nach beliebigen Kriterien.

An den Blättern sind üblicherweise Zeiger zu dem Zieldatensatz.

\section*{Mergesort}

\subsection*{2-Way-Mergesort}

\subsection*{2-Phase-Multiway-Mergesort}

\subsubsection*{Vorraussetzungen}

Größe der zu sortierenden Relation $= B$ Blöcke.

Anzahl der verfügbaren Seiten im Hauptspeicher $= M$.

$B \leq M^2$

\subsubsection*{Phase 1}

Relation wird blockweise in die $M$ Seiten im Hauptspeicher gelesen und dort sortiert.

Das Ergebnis wird anhand von sortierten Läufen der Länge $N$ in den Sekundärspeicher geschrieben.

\subsubsection*{Phase 2}

Sortierte Läufe werden gemischt.

Jeder Lauf wird sukzessive in eine der Seiten gelesen und in die Ausgabeseite durch Mischen sortiert ausgegeben.

Ist die Ausgabeseite voll wird sie in den Sekundärspeicher geschrieben.

\section*{Anfrageoptimierer}

\begin{itemize}
\item SQL Anfragen zu optimieren weil SQL nur das Resultat und nicht den Weg beschreibt.
\item Optimierung aufgrund von heuristischen Abschaetzungen.
\item Auch aufgrund von statistiken ueber die Datenbank.
\end{itemize}

\subsection*{Selectivitaet}

Anzahl der Resultate durch Gesamtanzahl der Datensaetze.

\section*{Datenbanktransaktionen}

Eine Datenbanktransaktion führt einen Konsistenten Zustand einer Datenbank in einen anderen Konsistenten Zustand über.

\subsection*{ACID}

\subsubsection*{Atomicity}

Die Überführung ist atomisch, d.h. wenn sie abgebrochen wird stellt sie den Originalen Zustand wieder her, sonst ist sie erfolgreich.

\subsubsection*{Consistency}

Die Transaktion hinterlässt, gegeben sie erhaelt einen konsistenten Zustand, immer einen konsistenten Zustand.

\subsubsection*{Isolation}

Transaktionen verhalten sich immer so, als würden sie sequenziell Ausgeführt werden.

\subsubsection*{Durability}

Wenn etwas schlimmes passiert, kann der Zustand immer wiederhergestellt werden.

\subsection*{Lost Update}

Zwei Transaktionen bearbeiten gleichzeitig die selbe Ressource.

Änderungen der ersten Transaktion kann direkt durch die zweite Transaktion überschrieben werden.

Lösung: Beim Ausführen eines Lese-/ Schreibzugriffs wird die genutzte Ressource für die Dauer der Transaktion gesperrt, damit andere Transaktionen die Ressource nicht ändern können.
\begin{itemize}
\item Share-Lock (Read-Lock): ermöglicht beliebig vielen Transaktionen einen Lesezugriff
\item Exclusive-Lock (Write-Lock): ermöglicht nur einer Transaktionen einen Schreibzugriff
\end{itemize}

\subsection*{Zweiphasensperrprotokoll}

\begin{enumerate}
\item Alle Ressourcen sperren
\item Damit arbeiten
\item Alle Ressourcen freigeben
\end{enumerate}

Im Gegensatz zu:

\begin{enumerate}
\item Erste Ressource holen
\item Damit arbeiten
\item Erste Ressource freigeben
\item Weiter arbeiten
\item Wenn nun ein Rollback passiert muessen evtl. andere Transaktion die inzwischen die erste Ressource benutzt haben auch einen Rollback machen! Das ist Mist!
\end{enumerate}

\subsection*{Deadlock Avoidance}

Transaktionen bekommen Startzeitstempel.

wait/die:

\begin{itemize}
\item älterer wartet
\item jüngerer wird neu gestartet (die) wenn er lock von älterem will (ergo kann der ältere im deadlockfall weiter laufen)
\end{itemize}

wound/wait:

\begin{itemize}
\item älterer schiesst jüngeren ab der seinen lock hält
\item jüngerer wartet auf älteren
\end{itemize}

In beiden Fällen wird der ältere immer zuerst fertig!

\subsection*{Shadow Paging}

Wenn eine Seite modifiziert werden soll, wird eine neue Seite (shadow page) allokiert, auf der gearbeitet wird.

Sind alle Änderungen vollständig, werden alle Referenzen auf das Original auf die 'shadow page' umgemappt.

\subsection*{Write Ahead Logging (WAL)}

Modifikationen müssen vor dem eigentlichen Durchführen protokolliert werden, sind daher im Fehlerfall widerholbar.

\end{document}
