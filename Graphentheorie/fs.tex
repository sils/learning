\documentclass[10pt,a4paper]{article}
\usepackage[left=2cm,right=2cm,top=2cm,bottom=2cm]{geometry}
\usepackage[utf8]{inputenc}
\usepackage[german]{babel}
\usepackage{amsmath}
\usepackage{amsfonts}
\usepackage{amssymb}
\usepackage{amsthm}
\newcommand{\abs}[1]{\ensuremath{\left\vert#1\right\vert}}
\begin{document}
\section{Allgemeines}
Ein \textbf{Graph} $G$ wird beschrieben durch $G = (V, E)$.

Ein Graph $K_n = (V, E)$ ist \textbf{vollständig} wenn alle möglichen Kanten für die Knoten V in E enthalten sind.

Ein Knoten $x$ hat den \textbf{Grad} $n =: deg(x)$ wenn er genau $n$ Nachbarn hat. Ist $deg(x) = 0$ so ist $x$ ein \textbf{isolierter Knoten}. Ist $deg(x) = 1$ so ist $x$ ein \textbf{Blatt}.

Ein Graph $G$ ist genau dann \textbf{zusammenhängend} wenn jeder Knoten direkt oder indirekt mit jedem anderen Verbunden ist.

Ein Graph $C_n = (V, E)$ ist ein \textbf{Kreisgraph} wenn alle Knoten den Grad zwei haben und der Graph zusammenhängend ist.

Eine \textbf{Clique} in einem Graphen $G$ ist ein Subgraph dieses Graphen der isomorph zu einem Graphen $K_n$ ist (für ein beliebiges $n$). Das größtmögliche $n =: \omega(G)$ ist die \textbf{Cliquenzahl}

Eine \textbf{Stabile Menge} in einem Graphen $G$ ist ein Subgraph dieses Graphen der isomorph zu einem Graphen $E_n$ ist (für beliebiges $n$). Das größtmögliche $n =: \alhpa(g)$ ist die \textbf{Stabilitätszahl}

Der Graph $E_n = (V, \emptyset)$ ist der \textbf{leere Graph}.

Ein \textbf{Matching} ist eine Auswahl an eindeutigen Zuordnungen von Elementen einer Menge.

Eine \textbf{Knotenüberdeckung (Vertex Cover)} für einen Graphen $G = (V, E)$ ist eine Menge $V' \subset V$ wobei es für jedes $e \in E$ ein $v \in V'$ gibt für das $v \in e$.
\section{Färbung}
\textbf{Chromatische Zahl} ist $\chi(G) := min\lbrace k \in \mathbb{N} : \mbox{G hat k Färbung} \rbrace$.

Für jeden Graphen $G(V,E)$ gilt $\omega(g) \leq \chi(g)$ und
$\frac{|V|}{\alpha(G)} \leq \chi(g)$.

Ein Graph $G$ ist \textbf{bipartit} wenn $\chi(G) \leq 2$.

Graph ist bipartit $\Leftrightarrow \nexists$ Kreis ungerader Länge in G
\subsection{Greedy}
Für alle Knoten $v \in V$ aus $G = (V, E)$: nehme ausschließlich bekannte Knotenfärbungszahlen aller verbundenen Knoten in eine Menge $N$, Knotenfärbung von v ist die kleinste Zahl aus den natürlichen Zahlen ohne $N$.
\section{Planarität}
Ein Graph ist genau dann planar, wenn er keine Unterteilung des K_5 oder des K_3,3 besitzt

Eulersche Polyederformel: ($R$ sei die Anzahl der Regionen inkl. Außenregion)
\[|V|-|E|+|R|=2\]
Weiterhin: $|E| \leq 3n -6$

$2|E| \leq 3|R|$
\section{Netzwerke}
\subsection{Minimaler Spannbaum}
Kruskal, Gewichte aufsteigend betrachten und nur inkludieren wenn neuer Knoten eingebunden wird oder Partitionen verbunden werden.
\subsection{Floyd, Dajkstra, Kruskal}
\subsubsection{Floyd, Kürzeste Pfade}
$d(i,j)$ initialisieren. Für alle $k \in 1 \mbox{ bis } n$: Für alle Knotenpaare $i, j$ sei $d(i, j) = min (d(i,j), d(i,k) + d(k,j))$.
\subsubsection{Dajkstra, Kürzeste Pfade bei nichtnegativen Kanten}
Nehme Knoten mit minimaler Distanz zum Startknoten - hinzufügen. (Nachbarknoten ggf. updaten.)
\subsubsection{Kruskal, Minimaler Spannbaum}
Aus unbenutzten Kanten die kürzeste wählen, die mit den gewählten keinen Kreis bildet. Wiederholen.
\section{Komplexitäten (vereinfacht)}
\subsection{Abschätzung nach oben}
$f \in O(g) \Leftrightarrow \lim_{x \rightarrow \infty} \abs{\frac{f(x)}{g(x)}} < \infty $ \,\,\,\,\,\,\,\,\,\,\,\,\,\,\,\,\,\,\,\,\,\,\,\,\,\,\,\, $f(n) = O(g(n)) \Leftrightarrow f(n) \leq c \cdot g(n)$
\subsection{Abschätzung}
\[f \in \Theta(g) \Leftrightarrow 0 <\lim_{x \rightarrow \infty} \abs{\frac{f(x)}{g(x)}} < \infty \]
\[f(n) = \Theta(g(n)) \Leftrightarrow c_1 \cdot g(n) \leq f(n) \leq c_2 \cdot g(n)\]
\subsection{Abschätzung nach unten}
$f \in \Omega(g) \Leftrightarrow 0 < \lim_{x \rightarrow \infty} \abs{\frac{f(x)}{g(x)}}$ \,\,\,\,\,\,\,\,\,\,\,\,\,\,\,\,\,\,\,\,\,\,\,\,\,\,\,\,\,\,\,\,\,\,\,\,\,\,\, $f(n) = \Omega(g(n)) \Leftrightarrow f(n) \geq c \cdot g(n)$
\section{Matroide}
TODO
\section{Linear Programming}
\subsection{LP}
\subsection{Dual Problem}
\section{SAT}
k-SAT mit $k \geq 3$ sind NP-schwer.
$SAT \leq 3-SAT \leq Clique$
\section{Misc}
$\sum\limits_{v \in V} deg(v) = 2 \cdot |E|$

Ungewichtetes Scheduling: Tasks mit größter Penalty auf ihre Deadline oder davor setzen, abwärts.
\end{document}
