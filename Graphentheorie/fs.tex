\documentclass[10pt,a4paper]{article}
\usepackage[utf8]{inputenc}
\usepackage[german]{babel}
\usepackage{amsmath}
\usepackage{amsfonts}
\usepackage{amssymb}
\usepackage{amsthm}
\newcommand{\abs}[1]{\ensuremath{\left\vert#1\right\vert}}
\begin{document}
\section{Allgemeines}
Ein Graph $G$ wird beschrieben durch $G = (V, E)$.

Ein Graph $K_n = (V, E)$ ist vollständig wenn alle möglichen Kanten für die Knoten V in E enthalten sind.

Ein Knoten hat den Grad $n$ wenn er genau $n$ Nachbarn hat.

Ein Graph $G$ ist genau dann zusammenhängend wenn jeder Knoten direkt oder indirekt mit jedem anderen Verbunden ist.

Ein Graph $C_n = (V, E)$ ist ein Kreisgraph wenn alle Knoten den Grad zwei haben und der Graph zusammenhängend ist.
\section{Färbung}
Die Chromatische Zahl ist die kleinste Zahl $\chi(G)$ mit der der Graph $G$ eine zulässige Knotenfärbung mit $\chi$ Farben besitzt.

\subsection{Greedy}
Für alle Knoten $v \in V$ aus $G = (V, E)$: nehme ausschließlich bekannte Knotenfärbungszahlen aller verbundenen Knoten in eine Menge $N$, Knotenfärbung von v ist die kleinste Zahl aus den natürlichen Zahlen ohne $N$.
\section{Planarität}
Eulersche Polyederformel: ($R$ sei die Anzahl der Regionen inkl. Außenregion)
\[|V|-|E|+|R|=2\]
Weiterhin:
\[|E| \leq 3n -6\]
\section{Netzwerke}
\subsection{Minimaler Spannbaum}
Kruskal, Gewichte aufsteigend betrachten und nur inkludieren wenn neuer Knoten eingebunden wird oder Partitionen verbunden werden.
\subsection{Floyd, Dajkstra, Kruskal}
TODO
\section{Komplexitäten (vereinfacht)}
\subsection{Abschätzung nach oben}
\[f \in O(g) \Leftrightarrow \lim_{x \rightarrow \infty} \abs{\frac{f(x)}{g(x)}} < \infty \]
\[f(n) = O(g(n)) \Leftrightarrow f(n) \leq c \cdot g(n)\]
\subsection{Abschätzung}
\[f \in \Theta(g) \Leftrightarrow 0 <\lim_{x \rightarrow \infty} \abs{\frac{f(x)}{g(x)}} < \infty \]
\[f(n) = \Theta(g(n)) \Leftrightarrow c_1 \cdot g(n) \leq f(n) \leq c_2 \cdot g(n)\]
\subsection{Abschätzung nach unten}
\[f \in \Omega(g) \Leftrightarrow 0 < \lim_{x \rightarrow \infty} \abs{\frac{f(x)}{g(x)}} \]
\[f(n) = \Omega(g(n)) \Leftrightarrow f(n) \geq c \cdot g(n)\]
\section{Matroide}
TODO
\section{Linear Programming}
\subsection{LP}
\subsection{Dual Problem}
\section{SAT}
k-SAT mit $k \geq 3$ sind NP-schwer.
$SAT \leq 3-SAT \leq Clique$
\section{Misc}
$\sum\limits_{v \in V} deg(v) = 2 \cdot |E|$
\end{document}