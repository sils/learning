\documentclass[12pt,a4paper]{article}
\usepackage[utf8]{inputenc}
\usepackage[german]{babel}
\usepackage[T1]{fontenc}
\usepackage{amsmath}
\usepackage{amsfonts}
\usepackage{amssymb}
\usepackage{graphicx}
\usepackage[left=1.5cm,right=1.5cm,top=1.5cm,bottom=1.5cm]{geometry}
\author{Maike Meier and Lasse Schuirmann}
\title{Messtechnik und Messdatenverarbeitung - Blatt 7}
\begin{document}
\maketitle

\section*{Aufgabe 1}
\subsection*{1.1}
Um eine Ablösung des Sensors zu detektieren, kann der $\chi^2$-verteilung Anpassungstest durchgeführt werden:

\begin{itemize}
\item Nullhypothese: Die Daten sind normalverteilt.
\item Alternativhypothese: Die Daten sind nicht normalverteilt.
\end{itemize}

Voraussetzung: Die Messwerte sind statistisch unabhängig.
\subsection*{1.2}
\scalebox{0.9}{
 \begin{tabular}{|r|c|c|c|c|c|c|c|c|}
 \hline
 Klasse & $\leq 35.0$ & $35.1 - 35.5$ & $35.6 - 36.0$ & $36.1 - 36.5$ & $36.6 - 37.0$ & $37.1 - 37.5$ & $37.6 - 38.0$ & $\geq 38.0$ \\
 \hline
 Anzahl $n_i$ & $4$ & $9$ & $16$ & $20$ & $16$ & $16$ & $6$ & $8$ \\
 \hline
 \end{tabular}
}

\subsection*{1.3}
\subsection*{1.4}
\subsection*{1.5}

\end{document}