\documentclass[12pt,a4paper]{article}
\usepackage[utf8]{inputenc}
\usepackage[german]{babel}
\usepackage[T1]{fontenc}
\usepackage{amsmath}
\usepackage{amsfonts}
\usepackage{amssymb}
\usepackage{graphicx}
\usepackage[left=3cm,right=3cm,top=3cm,bottom=3cm]{geometry}
\usepackage[table]{xcolor}
\usepackage{hyperref}
\setlength{\parskip}{3mm}
\setlength{\parindent}{0cm}
\DeclareMathOperator\erf{erf}
\author{Maike Meier und Lasse Schuirmann}
\title{Messtechnik und Messdatenverarbeitung - Kalman-Filter}
\newcommand*{\blankpage}{
  \vspace*{\fill}
  \begin{flushright}
  \tiny THIS PAGE INTENTIONALLY LEFT BLANK.
  \end{flushright}
  \pagebreak
}
\begin{document}
\rowcolors{2}{gray!25}{white}

\maketitle
\pagebreak

\blankpage

\section{Inertiale Navigation}
Es ist nicht sinnvoll die Flughöhe allein auf Basis der Beschleunigung zu berechnen, da wie in der Aufgabenstellung beschrieben zusätlich noch Inhomogenitäten des Treibstoffgemischs und Luftdruckschwankungen berücksichtigt werden müssen.

\section{Messunsicherheit}
Mit zunehmender Flughöhe kann der aktuelle Zustand stärker durch Aktionen und 
Ungenauigkeiten der Sensoren beeinflusst werden, dadurch steigt auch die 
Varianz der Messwerte und somit die Messungenauigkeit. Der Fehler wird also mit Aufintegriert.

\section{GPS Messung}
Die (ideale) GPS Messung mit der bekannten Standardabweichung von $\sigma_h = 2m$ ist nicht Höhenabhängig insbesondere da die aktuelle Messung nicht von vorherigen Abhängt.

\section{Kalman: Grundlagen}
Ein Kalman-Filter schätzt den Zustand eines Prozesses auf Kenntnis früherer Beobachtungen.
Betrachtet werden zeitdiskrete Prozesse der Form 
$x_t = A_t * x_(t-1) + B_t *u_t + eps_t$
Die Messwerte werden durch den (geschätzten) Zustand $z_t = C_t * x_t  \delta_t$ beschrieben.
$x_t$ bildet einen Zustandvektor zum Zeitpunkt t mit kontinuierlichen Komponenten.
$u_t$ bildet einen Aktionsvektor.
$A_t$, Systemmatrix, (nxn) beschreibt den idealen Zustandsübergang von $t$ nach $t+1$
$B_t$, Steuermatrix, (nxl) beschreibt den Zustandsübergang der durch die Aktion $u_t$ bewirkt wird
$C_t$, Messmatrix, (kxn) beschreibt die Abbildung von Zustand $x_t$ auf Beobachtung $z_t$
$eps_t$, $delta_t$ beschreiben das Rauschen des Prozesses als Zufallsvariablen, unabhängig und mit
Kovarianzen $R_t$ und $Q_t$ verteilt
Daraus ergibt sich ein Kreislauf aus Prädiktion und Korrektur der Messungen.

\section{TODO}
Durch den Kalman-Filter wird die Messunsicherheit mit zunehmender Flughöhe tendenziell eher nicht
weiter ansteigen, da immer wieder Korrekturen vorgenommen werden.


\end{document}
